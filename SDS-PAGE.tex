\part{SDS-PAGE}
\newpage
%\newcommand{\up}[1]{\textsuperscript{#1}} 
\setlength{\parindent}{0pt}
%\newcommand{\tab}{\hspace*{2em}}
\setcounter{secnumdepth}{5}
\setcounter{section}{0}
\renewcommand*{\theHsection}{sds.\the\value{section}}
 
%%%%%%%%%%%%%%%
\section{Introduction}
%%%%%%%%%%%%%%%
\underline{PAGE:}
Polyacrylamide gel electrophoresis (PAGE) is probably the most common analytical technique used to separate and characterize proteins. A solution of acrylamide and bisacrylamide is polymerized. Acrylamide alone forms linear polymers. The bisacrylamide introduces crosslinks between polyacrylamide chains. The 'pore size' is determined by the ratio of acrylamide to bisacrylamide, and by the concentration of acrylamide. A high ratio of bisacrylamide to acrylamide and a high acrylamide concentration cause low electrophoretic mobility. Polymerization of acrylamide and bisacrylamide monomers is induced by ammonium persulfate (APS), which spontaneously decomposes to form free radicals. TEMED, a free radical stabilizer, is generally included to promote polymerization.

\underline{SDS PAGE:}
Sodium dodecyl sulfate (SDS) is an amphipathic detergent. It has an anionic headgroup and a lipophilic tail. It binds non-covalently to proteins, with a stoichiometry of around one SDS molecule per two amino acids. SDS causes proteins to denature and dissassociate from each other (excluding covalent cross-linking). It also confers negative charge. In the presence of SDS, the intrinsic charge of a protein is masked. During SDS PAGE, all proteins migrate toward the anode (the positively charged electrode). SDS-treated proteins have very similar charge-to-mass ratios, and similar shapes. During PAGE, the rate of migration of SDS-treated proteins is effectively determined by molecular weight.
\section{Materials required}
	%%%%%%%%%%%%%%%
	\subsection{Chemicals/solutions}
	\begin{packed_enum}
	\item 30 \% Acrylamide/bisAcrylamide (37.5:1)
	\item 10\% SDS
	\item 1.5M Tris pH 8.8
	\item 10\% APS		
	\item	TEMED		
	\item 	DDH$_{2}$O	
	\end{packed_enum}
	
	\subsection{Equipment}
	\begin{packed_enum}
		\item Protein gel rig
		\item Gel doc
	\end{packed_enum}
 

%%%%%%%%%%%%%%%
\section{Protocol}
%%%%%%%%%%%%%%%
	\subsection{Acrylamide gel \%}
	%%%%%%%%%%%%%%%
	\begin{tabular}{l r r r r r}
	{\bf   } & {\bf 6\%} & {\bf 8\%} & {\bf 10\%} & {\bf 12\%} & {\bf 15\%}\\
			\hline
	Acrylamide	&	1.6 $\mu$l	& 2.1 $\mu$l & 2.7 $\mu$l & 3.2 $\mu$l &	4 $\mu$l \\
	10\% SDS		 & 0.1 $\mu$l	 & 0.1 $\mu$l & 	0.1 $\mu$l & 	0.1 $\mu$l	 & 0.1 $\mu$l\\
	1.5M Tris pH 8.8  & 2.6 $\mu$l	 & 2.6 $\mu$l	 & 2.6 $\mu$l	 & 2.6 $\mu$l	& 2.6 $\mu$l\\
	10\% APS		 & 0.1 $\mu$l	 & 0.1 $\mu$l	 & 0.1 $\mu$l	 & 0.1 $\mu$l	 & 0.1 $\mu$l\\
	TEMED		 & 0.01 $\mu$l & 0.01 $\mu$l & 0.01 $\mu$l	 & 0.01 $\mu$l	 & 0.01 $\mu$l\\
	H2O		 & 5.6 $\mu$l	 & 5.1 $\mu$l	 & 4.5 $\mu$l	 & 4.0 $\mu$l	 & 3.2 $\mu$l\\
		\hline
	
	\end{tabular}
%\end{document} % DONE WITH DOCUMENT!
