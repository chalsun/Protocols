 \part{MUG assay}

\newpage
%\newcommand{\up}[1]{\textsuperscript{#1}} 
\setlength{\parindent}{0pt}
%\newcommand{\tab}{\hspace*{2em}}
\setcounter{secnumdepth}{5}
\setcounter{section}{0}
\renewcommand*{\theHsection}{exy.\the\value{section}}
%%%%%%%%%%%%%%%
\section{Introduction}
The Fluorescent $\beta$-Galactosidase Assay (MUG) is a highly sensitive, fluorescent assay for determining the $\beta$-galactosidase activity in the lysates of cells transfected with a $\beta$-galactosidase expression construct
 
The study of the lac operon has played an important role in understanding the control of gene expression in bacteria. In prokaryotes, gene expression is controlled primarily at the level of transcription. For eukaryotes, the promoter activity can be analyzed by using fusion genes containing the promoter of interest attached to the bacterial $\beta$-galactosidase gene and be assayed by measuring $\beta$-galactosidase activity. Because β-galactosidase has a high turnover rate and is absent in mammalian cells it serves as a very useful and sensitive reporting tool for gene expression.
 
Esters of the fluorescent compound, 4-methylumbelliferone (4-MU), provide a sensitive, quantitative assay for $\beta$-galactosidase.  4-methylumbelliferyl-$\beta$-D-galactopyranoside (4-MUG) is a substrate of $\beta$-galactosidase that does not fluoresce until cleaved by the enzyme to generate the fluorophore 4-methylumbelliferone.  The assay can be used with extracts from different expression systems including mammalian, insect cells, yeast, and bacteria.
 
The Fluorescent $\beta$-Galactosidase Assay (MUG) provides a 96 well assay format for galactosidase activity that is suitable for high throughput applications.  The production of the fluorphore is monitored at an emission/excitation wavelength of 365/460nm.
\section{Materials}
	\subsection{Chemicals}
	\begin{packed_enum}
	\item 4-Methylumbelliferyl-{$\beta$}-D-glucuronide (MUG)
	\item 4-Methylumbelliferone (4-MU)
	\item Sodium carbonate (Na$_{2}$CO$_{3}$)
	\item Dimethylsulfoxide (DMSO)
	\end{packed_enum}
		
	\subsection{Equipment}
	\begin{packed_enum}
	\item Plate reader
	\item 37 {\textcelsius} incubator
	\end{packed_enum}
 
%%%%%%%%%%%%%%%
\section{Protocol}
%%%%%%%%%%%%%%%
	\subsection{Preparation of stock solutions}
	\begin{packed_enum}
	\item {\underline{2mM MUG}}. Prepare MUG substrate stock solution by weighing out 7 mg MUG in a microtube. Add 40 $\mu$l DMSO to dissolve. Transfer to a 15 ml falcon tube and bring the final volume to 10 ml with ddH$_{2}$O.
	\item \underline{1 mM 4-MU}. Prepare by weighing out 1.7 mg 4-MU in a microtube. Add 500 $\mu$l DMSO to dissolve. Transfer to a 15 ml falcon tube and bring up to a final volume of 10 ml in extraction buffer. Wrap in foil and store in 4\textcelsius\ for a maximum of one month.
	\end{packed_enum}
	%%%%%%%%%%%%%%%
	\subsection{Preparation of working solutions}
	\begin{packed_enum}
		\item {\underline{10$\mu$M 4-MU}}. Prepare enough for use on the day of assay by diluting 4mM stock in extraction buffer.
		\item {\underline{Stop buffer}}. Dissolve 21.2 g Na$_{2}$CO$_{3}$ in ddH$_{2}$O and make up to 1 L. pH 10.5
	\end{packed_enum}
\section{Procedure}
	

\end{document} % DONE WITH DOCUMENT!
